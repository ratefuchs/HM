\documentclass{scrartcl}
\usepackage[T1]{fontenc}
\usepackage[utf8]{inputenc}
\usepackage[ngerman]{babel}
\usepackage{amsmath}
\usepackage{amsfonts}

\begin{document}
\section*{Aufgabe 1}
Lösungsskizze (Zwischenergebnisse):\\
Das charakteristische Polynom ist $(\lambda-4)^2(\lambda-i)(\lambda+i)\Rightarrow \lambda_{1,2}=4, \lambda_3=i, \lambda_4=-i=\bar{i}$.\\
$\text{Kern}(A-4I)=\left[\begin{pmatrix}1\\-1\\0\\0\end{pmatrix}\right]$\\
$\text{Kern}((A-4I)^2)=\left[\begin{pmatrix}1\\0\\0\\0\end{pmatrix},\begin{pmatrix}0\\1\\0\\0\end{pmatrix}\right]$\\
$\text{Kern}(A-i\cdot I)=\left[\begin{pmatrix}0\\0\\1+i\\-1\end{pmatrix}\right]$\\
Als (ein mögliches) Fundamentalsystem ergibt sich
$$\left\lbrace e^{4x}\begin{pmatrix}1\\-1\\0\\0\end{pmatrix}, e^{4x}\begin{pmatrix}1-x\\x\\0\\0\end{pmatrix},
\begin{pmatrix}0\\0\\ \cos x-\sin x\\-\cos x\end{pmatrix},
\begin{pmatrix}0\\0\\ \cos x+\sin x\\-\sin x\end{pmatrix}\right\rbrace$$

\section*{Aufgabe 2}
Lösungsskizze (Zwischenergebnisse):\\
(a) $A$ ist schon in Jordan-Normalform. Damit lässt sich das Fundamentalsystem direkt ablesen als
$$\left\lbrace e^{2x}\begin{pmatrix}1\\x\\0\end{pmatrix}, e^{2x}\begin{pmatrix}0\\1\\0\end{pmatrix},
e^{-x}\begin{pmatrix}0\\0\\ 1\end{pmatrix}\right\rbrace$$
Inhomogene Gleichung:\\
$c^\prime(x)=(Y(x))^{-1}\cdot b(x)  =\begin{pmatrix}1\\-x\\ (1+x)e^x\end{pmatrix}\\
\Rightarrow c(x)= \begin{pmatrix}x\\-\frac{1}{2}x^2\\ xe^x\end{pmatrix}
\Rightarrow y_p(x)=Y(x)\cdot c(x)=\begin{pmatrix}xe^{2x}\\\frac{1}{2}x^2e^{2x}\\ x\end{pmatrix}$.\\
Als allgemeine Lösung ergibt sich $$c_1e^{2x}\begin{pmatrix}1\\x\\0\end{pmatrix}+c_2e^{2x}\begin{pmatrix}0\\1\\0\end{pmatrix} + c_3e^{-x} \begin{pmatrix}0\\0\\ 1\end{pmatrix}+\begin{pmatrix}xe^{2x}\\\frac{1}{2}x^2e^{2x}\\ x\end{pmatrix}\quad (c_1,c_2,c_3\in\mathbb{R})$$\newline
(b) Es ergibt sich $c_1=1, c_2=0, c_3=2$.
\end{document}