\documentclass{scrartcl}
\usepackage[T1]{fontenc}
\usepackage[latin1]{inputenc}
\usepackage[ngerman]{babel}
\usepackage{amsmath}
\usepackage{amsfonts}

\begin{document}
\section*{Aufgabe 1}

Berechne jeweils $(g\circ f)^\prime$ direkt durch ausrechnen und differenzieren und nach der Kettenregel.\\
(a) $f:\mathbb{R}^3\rightarrow\mathbb{R}^2, f(x,y,z)=(x+y,y+2z)$\\
$g:\mathbb{R}^2\rightarrow\mathbb{R}, g(t,u)=tu$\\
(b) $f:\mathbb{R}^3\rightarrow\mathbb{R}^4, f(x,y,z)=(y,x+z,z^2,xyz)$\\
$g:\mathbb{R}^4\rightarrow\mathbb{R}^2, g(x,y,z,w)=(x+y+w,zw)$\\
(c) $f:\mathbb{R}^3\rightarrow\mathbb{R}^2, f(x,y,z)=(e^x,z\sin y)$\\
$g:\mathbb{R}^2\rightarrow\mathbb{R}^4, g(t,u)=(t,t^2u,\sin t,(\log u)+t)$\\
Bemerkung: Es muss hier nicht explizit \"uberpr\"uft werden, 
in welchen Bereichen die Funktion bzw. deren Ableitung definiert ist.\\
(d) $f:\mathbb{R}^2\rightarrow\mathbb{R}, f(x,y)=xy^2$\\
$g:\mathbb{R}\rightarrow\mathbb{R}^2, g(t)=(2t,\sin t)$\\
(e) Berechne ebenso $(f\circ g)^\prime$ mit f,g aus (d).\\
\end{document}