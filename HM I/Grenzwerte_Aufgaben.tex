\documentclass{scrartcl}
\usepackage[T1]{fontenc}
\usepackage[latin1]{inputenc}
\usepackage[ngerman]{babel}
\usepackage{amsmath}
\usepackage{amsfonts}

\begin{document}
\section*{Aufgabe 1}
Es sei $f:\mathbb{R}\rightarrow\mathbb{R}$ stetig differenzierbar und $f^\prime(0)=f^\prime(2)=3$.\\
(a) Sei $f(0)=0$. Berechne $\lim\limits_{x\rightarrow0^+}\frac{f(x+2)-4}{f(x)}$ in Abh\"angigkeit von $f(2)$!\\
(b) Sei $f(2)=4$. Finde eine Funktion f, so dass obiger Grenzwert nicht 1 ergibt. (Hinweis: $f^\prime(0)=f^\prime(2)$!)


\section*{Aufgabe 2}
Berechne folgende Grenzwerte:\\[0.5em]
(a) $\lim\limits_{x\rightarrow0}\frac{e^{\sin x}-1}{x}$\\[0.5em]
(b) $\lim\limits_{x\rightarrow\infty}\frac{e^x}{\sinh x}$\\[0.5em]
(c) $\lim\limits_{x\rightarrow0}\frac{\sinh x}{e^x}$
\end{document}