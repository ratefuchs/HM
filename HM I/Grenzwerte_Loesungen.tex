\documentclass{scrartcl}
\usepackage[T1]{fontenc}
\usepackage[latin1]{inputenc}
\usepackage[ngerman]{babel}
\usepackage{amsmath}
\usepackage{amsfonts}

\begin{document}
\section*{Aufgabe 1}
(a) f ist stetig differenzierbar und $f^\prime(0)=0$. Also ist f auf einem Intervall $[0,\epsilon)$ 
von 0 streng monoton wachsend, daher ist f(x)>0 ($0<x<\epsilon$).\\
Fall 1: f(2) > 4. Dann ist $f(x+2)-4$ in einer Umgebung von 0 positiv. Der Nenner des Bruchs geht gegen 0 (wobei er positiv ist) 
und der Z�hler bleibt gr��er 0. Damit ist $\lim\limits_{x\rightarrow0^+}\frac{f(x+2)-4}{f(x)}=\infty$.\\
Fall 2: f(2) < 4. Dann ist $f(x+2)-4$ in einer Umgebung von 0 negativ. Der Nenner des Bruchs geht gegen 0 (wobei er positiv ist) 
und der Z�hler bleibt kleiner 0. Damit ist $\lim\limits_{x\rightarrow0^+}\frac{f(x+2)-4}{f(x)}=-\infty$.\\
Fall 3: f(2) = 4. Dann liefert l'Hospital:
\[\lim\limits_{x\rightarrow0^+}\frac{f(x+2)-4}{f(x)}\stackrel{[\frac 0 0]}{=}
\lim\limits_{x\rightarrow0^+}\frac{f^\prime(x+2)}{f^\prime(x)}=\frac 3 3=1\].\\
(b) Sei f die lineare Funktion $f(x)=3x-2$. Dann ist $\frac{f(2)-4}{f(0)}=\frac{0}{-2}=0\\
\Rightarrow\lim\limits_{x\rightarrow0^+}\frac{f(x+2)-4}{f(x)}=0$.


\section*{Aufgabe 2}
Berechne folgende Grenzwerte:\\
(a) L'Hospital liefert $\lim\limits_{x\rightarrow0}\frac{e^{\sin x}-1}{\sin x}
\stackrel{[\frac 0 0]}{=}\frac{\cos xe^{\sin x}}{\cos x}=1.$\\
(b) Elementare Umformungen liefern $\lim\limits_{x\rightarrow\infty}\frac{e^x}{\sinh x}=
\lim\limits_{x\rightarrow\infty}\frac{e^x}{\frac{e^x-e^{-x}}{2}}=
\lim\limits_{x\rightarrow\infty}\frac{2}{1-e^{-2x}}=2.$\\[0.2em]
L'Hospital liefert nach zweimaligem Anwenden wieder die Ursprungsfunktion!\\
(c) $\lim\limits_{x\rightarrow0}\frac{\sinh x}{e^x}=\frac{\sinh 0}{e^0}=0/1=0.$ Hier darf L'Hospital NICHT angewendet werden!
\end{document}